@@Overview
Welcome to the RAGE ATL module.

* Origin Of Module Name *
ATL stands for "Another Template Library". 

* Overview *

ATL contains a
collection of templates that implement common containers and
some other useful data structures. Its original intent was to
replicate many of the functions of the C++ Standard Template
Library (STL), while avoiding some of the pitfalls of early
STL implementations.



There are advantages and disadvantages of using ATL versus
STL versus other approaches, so please choose carefully.



ATL containers generally store template nodes in them, rather
than storing the objects directly. Actually, STL works the
same way but hides this "feature" from the user, while ATL
exposes it. This means that instead of pushing and popping
instances of Object into your atSList\<Object\>, you'll be
pushing and popping instances of atSNode\<Object\>. With STL
you can push and pop Object instances directly, but you need
to use an iterator to get at items. ATL generally does not
require the use of iterators, so accessing items is a little
simpler.

@@Components
Here's a table listing the primary components of this module.

@@Relationships
The ATL module mostly only depends on the core library for
things like Asserts, Debug messages, and data types.



A couple header files in ATL depend on files in vector and
data. However, for the most part these are in template
declarations that have no corresponding .cpp files, so
linking with ATL will not automatically require linking with
these modules. Now, if an instance of these templated classes
gets made you'll have to link with vector and data.



ATL is used by many modules, so you'll generally have to link
with it whether you want to or not.

##BEGIN COMMAND-LINE-SECTION is automatically generated - do not edit this section!
@@Command Line Options
<TABLE>
Parameter Name  In File      Description                                                                                  
--------------  -----------  -------------------------------------------------------------------------------------------  
fastonly        qaitems.cpp  will only run fast tests allowing for fast tests locally and more stringent tests offline    
filename        testmap.cpp  name of file to do word counts on                                                            
qadraw          qaitems.cpp  Draw drawable qa items                                                                       
resultsfile     qaitems.cpp  The location of the file to put the results in (defaulting to qaresults).                    
runname         qaitems.cpp  The name to apply to the current run in the log file (defaults to the RAGE release number).  
showerrors      qaitems.cpp  outputs fails as errors                                                                      
testname        qaitems.cpp  Only run the test specified in this parameter.                                               
xmlformat       qaitems.cpp  saves results in xml format                                                                  
xmllog          qaitems.cpp  outputs in xml log                                                                           
</TABLE>
##END COMMAND-LINE-SECTION is automatically generated - do not edit this section!

@@Working Examples
##BEGIN SAMPLES-SECTION is automatically generated - do not edit this section!
##END SAMPLES-SECTION is automatically generated - do not edit this section!
