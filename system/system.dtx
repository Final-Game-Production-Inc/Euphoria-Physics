
@@Overview
* Origin Of Module Name *
System stands for "SYSTEM management".

* Overview *
The system module is a catch-all area for low-level functionality such as command line parsing, inter-process communication, cache management, time management (both at the simulation level and the profiling level), and program startup.

* Command Line Support *

All command line access should go through the sysParam class and the PARAM macros.

There are plenty of examples of proper usage throughout the rage code.

One unusual feature of our command line parsing is that options are compared such that a trailing 's' is ignored, no -nocpv and -nocpvs are considered equivalent.

On Xenon, the startup code always saves the last command line out to D:\commandline.txt.  If a xenon application is ever launched with
no command line, the code will look for a D:\commandline.txt file and use that for the next run.  This allows you to launch any application
from the xenon launcher however it was most recently launched from the debugger or xbreboot tool.  This feature is now disabled because
people never bothered to read this documentation and were confused by this behavior.

@@codecheck
<TITLE Code Correctness>
codecheck.h contains useful functions for enforcing code correctness and dealing with ignorable asserts at runtime. It is built on the
idea of three main points in error handling.
* Game error handling is achieved through error codes and return values
* Game Asserts are ignoreable
* Release build does not required Error handling, so remove it all.

* Helping with Error Codes and Return Values *
AssertRetVal and RETURNCHECK are useful for checking the return values. AssertRetVal will check that
a value returned is a success and pass it on to a function. RETURNCHECK forces a return value to be checked
by the user. Both work well together.

* Helping with ignoreable asserts *
If_Assert, If_NotAssert and AssertReturn deal with the case of ignoreable assert. They basically allow error handling code to be simply added
after an assert has occurred. If asserts are directly followed by a script block.

EXAMPLE
<B>Before</B>
<CODE>
	bool Func( Object* obj )
	{
		Object* obj2 = GetClosestObj( obj );
		
		obj += obj2->Smelliness()->Amount
		return true;
	}
</CODE>
<B>After</B>
<CODE>
	RETURNCHECK( bool ) Func( Object* obj )		// Force user to check return value
	{
		AssertReturn( obj );					// Check if null and return if so
		Object* obj2 = AssertRetVal( GetClosestObj( obj ) );		// Check return value of object
		
		IF_Assert( obj2 && obj2->Smelliness() )			// only do instruction if assert passed
		{
			obj += obj2->Smelliness()->Amount
		}
		return true;
	}
</CODE>


@@Components
Here's a table listing the primary components of this module.

@@Relationships
Insert text with inheritance and dependency info for the
classes in this library.  Ideally, include UML diagrams
(as a GIF) followed by any additional explanations.

##BEGIN COMMAND-LINE-SECTION is automatically generated - do not edit this section!
@@Command Line Options
<TABLE>
Parameter Name      In File            Description                                                                                                               
------------------  -----------------  ------------------------------------------------------------------------------------------------------------------------  
breakonaddr         main.cpp           Break on alloc address (in hex; stop when operator new would have returned this address)                                  
breakonalloc        main.cpp           Break on alloc ordinal value - value in {braces} in debug output (use =0 first to stabilize allocations)                  
forcebootcd         bootmgr_xenon.cpp  Force booting from cd                                                                                                     
forcebootcd         bootmgr_psn.cpp    Force booting from cd                                                                                                     
forcebootcd         bootmgr_win32.cpp  Force booting from cd                                                                                                     
forceexceptions     main.cpp           Force exceptions even when we think a debugger is present                                                                 
jobtimings          task_psn.cpp       Spew job timings                                                                                                          
localhost           main.cpp           Specify local host of PC for rag connection (automatically set by psnrun.exe)                                             
logallocstack       main.cpp           Log all memory allocation stack tracebacks (requires -logmemory)                                                          
logmemory           main.cpp           Log all memory allocation and deallocations to file (default is c:\\logmemory.csv)                                        
pedantic            main.cpp           Set the global pedantic level; zero to press onward, larger numbers (better!) let fewer questionable things slip through  
qadraw              qaitems.cpp        Draw drawable qa items                                                                                                    
rageversion         main.cpp           Display rage version this was compiled against, and then exit.                                                            
runname             qaitems.cpp        The name to apply to the current run in the log file (defaults to the RAGE release number).                               
testname            qaitems.cpp        Only run the test specified in this parameter.                                                                            
trackerfile         main.cpp           Log all tracking activity to named file                                                                                   
trackerfileverbose  main.cpp           Save allocation info with each tally                                                                                      
usethreadnames      ipc.cpp            Set thread names (causes problems with xbWatson)                                                                          
</TABLE>
##END COMMAND-LINE-SECTION is automatically generated - do not edit this section!

@@Working Examples
##BEGIN SAMPLES-SECTION is automatically generated - do not edit this section!
@@Semaphores Sample
<GROUP Working Examples>

This sample shows how to create semaphores and use them to communicate between threads
The sample is broken into several steps:
1. Declare our thread function
2. Create our thread

* Declare our thread function *
We declare the function that will be run in our separate thread.
<CODE>

DECLARE_THREAD_FUNC(foobar) {
ptr = 0;
Displayf("foobar: Polling on s_Sema...");
if (sysIpcPollSema(s_Sema)) {
Displayf("foobar: got it!");
}
else {
Displayf("foobar: Poll failed, blocking now");
sysIpcWaitSema(s_Sema);
}

Displayf("foobar: have the sema now, sleeping for a bit");
sysIpcSleep(5000);

Displayf("foobar: releasing sema");
sysIpcSignalSema(s_Sema);

Displayf("foobar: done");
}
</CODE>

* Create our thread *
We create the thread.  This thread runs the function <c>foobar(void* ptr)</c>, with a stack size of 4 KB, and the thread is suspended (not started.)
<CODE>

sysIpcCreateThread(foobar,NULL,4096,PRIO_NORMAL,"foobar",true);
</CODE>


##END SAMPLES-SECTION is automatically generated - do not edit this section!
